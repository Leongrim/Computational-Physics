% !TeX root = ../Notizen.tex
\subsection*{a)}
Zur numerischen Berechnung des Hauptwertintegrals wird die folgende Formel
aus der Vorlesung angewendet:
\begin{align}
I_1=\mathcal{P}\int\limits_{-1}^{1}\frac{\exp(t)}{t}\text{d}t=
\int\limits_{-1}^{-\Delta}\frac{\exp(t)}{t}\text{d}t+
\int\limits_{\Delta}^{1}\frac{\exp(t)}{t}\text{d}t+
\int\limits_{-1}^{1}\frac{\exp(\Delta\cdot t)-1}{t}\text{d}t
\end{align}

\subsection*{b)}
Da die Auswertung des Integranden Singularitäten an den Grenzen aufweist,
wird eine partielle Integration durchgeführt, die dieses Problem behebt.
\begin{align}
I_2=\int\limits_{0}^{\infty}\frac{\exp(-t)}{\sqrt{t}}\text{d} t=
\underbrace{\left\exp(-t)\cdot\frac{1}{-\frac{1}{2}+1}\cdot\sqrt{t}\right
\vert_{0}^{\infty}}_{=0}-\int\limits_{0}^{\infty}-\exp(-t)\cdot
 2\sqrt{t}\text{d}t
\end{align}
Analytisch ergibt sich die Gammafunktion:
\begin{align}
I_2=\int\limits_{0}^{\infty}\frac{\exp(-t)}{\sqrt{t}}\text{d} t=
\Gamma\left(\frac{1}{2}\right)=\sqrt{\pi}
\end{align}
mithilfe des Ergänzungssatzes
$\Gamma(x)\cdot\Gamma(1-x)=\frac{\pi}{\sin(\pi x)}$ mit $x=\frac{1}{2}$.
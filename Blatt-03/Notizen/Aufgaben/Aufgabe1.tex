% !TeX root = ../Notizen.tex

\section*{Aufgabe1: Runge-Kutta 4. Ordnung}
In dieser Aufgabe soll das Runge-Kutta Verfahren in 4. Ordnung
für ein beliebiges Kraftfeld implementiert werden.
Dabei soll gelten:
\begin{align}
	\dot{\vec{r}}=&\vec{v}\\
	\dot{\vec{v}}=&\frac{1}{m}\vec{F}(\vec{r}).
\end{align}
Das Runge-Kutta Verfahren löst Differentialgleichungssysteme 1. Ordnung. Dazu wird das Problem umgeschrieben in
\begin{align}
	\vec{y}&\equiv
	\begin{pmatrix}
	y_1=y\\ y_2\\ \vdots\\y_{n-1}=y^{(n-1)}\\ y_n=y^{(n)}
	\end{pmatrix}=
	\begin{pmatrix}
		y_2\\y_3\\\vdots\\y_n\\f(x,y_1,y_2,\cdots,y_n)
	\end{pmatrix}
	\equiv\vec{f}(x,\vec{y}).
\end{align}
In dem vorliegenden Problem wird mit
\begin{align}
	\dot{\vec{y}}=\begin{pmatrix}
		\dot{\vec{r}}\\\dot{\vec{v}}
	\end{pmatrix}
	=
	\begin{pmatrix}
	\vec{v}\\\vec{F}(\vec{r})	
	\end{pmatrix}
	=\vec{f}(t,\vec{y})
\end{align}
gearbeitet.
Die Masse $m$ wird auf 1 gesetzt.
Für diskrete Zeiten $T= N\cdot h$ löst das Verfahren das Gleichungssystem mithilfe der Trapezregel zur Integration.
Daraus entsteht die Regression
\begin{align}
	\vec{k}_1&=h\cdot \vec{f}(t_n ,~\vec{y}_n)\\
	\vec{k}_2&=h\cdot \vec{f}\left(t_{n+^1\!/\!_2},~\vec{y}_n+\frac{1}{2}\vec{k}_1\right)\\
	\vec{k}_3&=h\cdot \vec{f}\left(t_{n+^1\!/\!_2},~\vec{y}_n+\frac{1}{2}\vec{k}_2\right)\\
	\vec{k}_4&=h\cdot \vec{f}\left(t_{n+1},~\vec{y}_n+\vec{k}_3\right)\\
	\vec{y}_{n+1}&=\vec{y}_n + \frac{1}{6}\left[ \vec{k}_1 + 2 \vec{k}_2 +2\vec{k}_3 +\vec{k}_4 \right]
\end{align} 
mit dem Startwert
\begin{align}
	\vec{y}_0=
	\begin{pmatrix}
		\vec{r}_0\\\vec{v}_0
	\end{pmatrix}.
\end{align}
Der Funktion zur Lösung der Differentialgleichung wird eine Endzeit übergeben, die Startvektoren $\vec{r}_0$ und $\vec{v}_0$ sowie die Intervallbreite $h$. Gegebenfalls kann ihr noch eine Datei übergeben werden, in die die Werte eingespeichert werden, andernfalls gibt sie einen Vektor aus
\begin{align}
	y_N = 
	\begin{pmatrix}
		\vec{r}_N\\\vec{v}_N\\ t_N
	\end{pmatrix}.
\end{align}

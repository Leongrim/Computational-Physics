% !TeX root = ../Notizen.tex

\section*{Aufgabe1: Runge-Kutta 4. Ordnung}
In dieser Aufgabe soll das Runge-Kutta Verfahren in 4. Ordnung implementiert werden, für ein beliebiges Kraftfeld.
Dabei soll gelten
\begin{align}
	\dot{\vec{r}}=&\vec{v}\\
	\dot{\vec{v}}=&\frac{1}{m}\vec{F}(\vec{r}).
\end{align}
Das Runge-Kutta Verfahren ist löst Differentialgleichungssysteme Erste Ordnung. Dazu wird das Problem umgeschrieben in
\begin{align}
	\vec{y}&\equiv
	\begin{pmatrix}
	y_1=y\\ y_2\\ \vdots\\y_{n-1}=y^{(n-1)}\\ y_n=y^{(n)}
	\end{pmatrix}=
	\begin{pmatrix}
		y_2\\y_3\\\vdots\\y_n\\f(x,y_1,y_2,\cdots,y_n)
	\end{pmatrix}
	\equiv\vec{f}(x,\vec{y}).
\end{align}
In dem Vorliegenden Problem wird mit
\begin{align}
	\dot{\vec{y}}=\begin{pmatrix}
		\dot{\vec{r}}\\\dot{\vec{v}}
	\end{pmatrix}
	=
	\begin{pmatrix}
	\vec{v}\\\vec{F}(\vec{r})	
	\end{pmatrix}
	=\vec{f}(t,\vec{y})
\end{align}
gearbeitet.

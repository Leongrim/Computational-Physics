% !TeX root = ../Notizen.tex

\section*{Aufgabe 1: Bifurkationsdiagramme}
	In dieser Aufgabe, werden die Bifurkationsdiagramme für zwei unterschiedliche Abbildungen erstellt.
	\begin{align}
		&\text{logistische Abbildung: }x_{n+1}=rx_n(1-x_n)\text{ wobei gilt }x_n\in[0,1]\label{eq:logistik}\\
		&\text{kubische Abbildung: }x_{n+1}=rx_n-x^3_n\text{ wobei gilt }x_n\in[-\sqrt{1+r},\sqrt{1+r}]
	\end{align}
	Durch diese Iteration lassen sich Fixpunkte $f(x^*)=x^*$ finden.
	Für $r>0$ gibt es nur einen Fixpunkt $x^*=0$
	und für $r>1$ergibt sich ein weiterer Punkt. 
	Interessant wird das dieses Verfahren für große $r$ keine Fixpunkte mehr findet sondern nur noch Orbits, auf denen um den Fixpunkt oszilliert wird.
	Ein Beispiel ist in \cref{fig:Orbit}.
	Mit weiter anwachsenden $r$ nimmt die zahl der Orbits immer weiter zu.
	Jeder Orbit $x_n$ ist ein Teil des Bifurkationsdiagramms.
	Dazu wird eine Funktion geschrieben, der eine Anzahl an Iterationen über geben wird, und wie viele Werte von aus Gespeichert werden sollen.
	Dieser Ansatz kann verwendet werden unter der Annahme das die Anzahl der Iterationen groß genug ist, dass sich der Algorithmus schon in der Oszilation befindet
	\begin{figure}
		\centering
		\includegraphics[width = 0.5\textwidth]{../Plots/Aufgabe1a/Plot_70.png}\caption{Ein Beispiel Plot für ein Orbit.}\label{fig:Orbit}
	\end{figure}
	\begin{figure}
		\centering
		\includegraphics[width = \textwidth]{../Plots/Plot_1_a.png}
		\caption{Hier ist das Ergebnis für die logistische Abbildung dargestellt.}
	\end{figure}
	\begin{figure}
		\centering
		\includegraphics[width = \textwidth]{../Plots/Plot_1_b.png}
		\caption{Hier ist das Ergebnis für die kubische Abbildung dargestellt.}
	\end{figure}
	Für die Ergebnisse wurden 1000 Iterationsschritte durchgeführt und die letzten 500 werden gespeichert.
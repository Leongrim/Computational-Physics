% !TeX root = ../Notizen.tex

\section*{Aufgabe 1: Monte-Carlo-Simulation eines einzelnen Spins}
Es soll mit Hilfe des Metropolis-Algorithmus ein einzelner Spin $s=\pm1$ in einem äußeren Magnetfeld $H$ mit der Energie \[\mathcal{H}=-sH\] simuliert werden.
Dem Metropolis-Algorithmus soll ein Spin-Flip $s\rightarrow-s$ angeboten werden.
\subsection*{a)}
Zunächst soll analytisch der Wert der Magnetisierung \[m=\braket{s}\] als Funktion des Magnetfeldes $H$ berechnet werden.
Dazu wird zunächst die Zustandssumme $Z$ bestimmt: \[Z=\sum_{i}\mathrm{e}^{-\beta\mathcal{H}(i)}=\sum_{i}\mathrm{e}^{\beta s(i)H}=\frac{1}{2}\left(\mathrm{e}^{\beta H}+\mathrm{e}^{-\beta H}\right)=\cosh(\beta H),~~~\text{mit}~\beta=\frac{1}{k_\text{B}T}.\]
Der Ensemble-Mittelwert der Spin-Variablen $s$ ist
\begin{align*}\braket{s}&=\sum_{i}\frac{1}{Z}\mathrm{e}^{-\beta\mathcal{H}(i)}s(i)=\sum_{i}\frac{1}{\cosh(\beta H)}\mathrm{e}^{\beta s(i)H}s(i)=\frac{1}{\cosh(\beta H)}\cdot\frac{1}{2}\left(\mathrm{e}^{\beta H}-\mathrm{e}^{-\beta H}\right)\\
&=\frac{\sinh(\beta H)}{\cosh(\beta H)}=\tanh(\beta H)=m(H).\end{align*}

\subsection*{b)}
Nun soll numerisch die Magnetisierung $m$ berechnet werden, wobei $\beta=1$ gesetzt wird.
Dazu werden die Monte-Carlo-Simulationen mit $\num{e4}$ Werten $H\in[-5,\,5]$ mit je $\num{e5}$ Zeitschritten ausgeführt.
Das Ergebnis ist in \cref{fig:Magnetisierung} zu sehen.\newpage

Das Verfahren funktioniert folgendermaßen:
\begin{enumerate}
\setcounter{enumi}{-1}
\item Initialisierung: Zur Zeit $t$ befinde sich das System im Zustand $i(t)=i$, hier Spin $s=+1$.
\item Biete MC-move $i\rightarrow j$ an, hier $s=+1\rightarrow s'=-1$.
\item Akzeptiere vorgeschlagenen Move:
\begin{enumerate}
\item Berechne $\Delta E=\mathcal{H}(j)-\mathcal{H}(i)$, hier $\Delta E=\mathcal{H}(s')-\mathcal{H}(s)=2sH$.
\item Immer akzeptieren, wenn $\Delta E<0$.
\item Wenn $\Delta E>0$, ziehe gleichverteilte Zufallszahl $p\in[0,\,1]$ und berechne $\mathrm{e}^{-\beta\Delta E}$. Akzeptieren, wenn $p<\mathrm{e}^{-\beta\Delta E}$, sonst ablehnen.
\end{enumerate}
\item Akzeptieren $\rightarrow$ neuer Zustand $i(t+\Delta t)=j$\\Ablehnen $\rightarrow$ neuer Zustand $i(t+\Delta t)=i=$ alter Zustand
\end{enumerate}
Danach wird in einer Zeitschleife wieder zu 1. gesprungen. Während der Schleife (jeweils nach 3.) wird gemessen:
\begin{enumerate}[resume]
\item Messung: MC-Mittel über gesamplete Zustände (nach $t_\text{MC}$ MC-Schritten)
\[\braket{O}_\text{MC}=\frac{1}{t_\text{MC}}\sum_{n=1}^{t_\text{MC}}O(i(n\Delta t))=\braket{O},~~~\text{hier }O(i)=s~\text{und }\braket{s}=m.\]
\end{enumerate}

\begin{figure}[H]
\centering
\includegraphics[width=\textwidth]{../Plots/1_B_Magnetisierung.pdf}
\caption{Magnetisierung $m$ in Abhängigkeit vom Magnetfeld $H$.}
\label{fig:Magnetisierung}
\end{figure}
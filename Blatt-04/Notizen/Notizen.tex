\documentclass[a4paper, 12pt]{article}
\linespread{1.25}
\usepackage{ngerman}
\usepackage[utf8]{inputenc}
\usepackage{amsmath}
\usepackage{amssymb}
\usepackage{dsfont}
\usepackage{graphicx}
\usepackage{subfigure}
\usepackage{float}
\usepackage{anysize}
\usepackage{icomma}
\usepackage[german]{cleveref}
\usepackage[left=2.5cm, right=2.5cm, top=2cm, bottom=2cm]{geometry}
\usepackage{listings}
\usepackage{siunitx}
\usepackage{textcomp}
%\usepackage{caption}
%\usepackage{subfigure}
\crefname{figure}{Abbildung}{Abbildungen}
\crefname{subfigure}{Abbildung}{Abbildungen}

\begin{document}
\title{
\textbf{Computatinal Physics\\
Übung}
}
\date{}
\maketitle

\begin{center}
Jan Doersch, Sebastian Jäger und Leonard Wollenberg
\end{center}
In Programm 1 werden die Daten für die Aufgabe 1, Aufgabe 2 a) und b), für das Doppelpendel mithilfe des Runge-Kutta Verfahrens 4ter Ordnung.
In Programm 2 werden die Poincaré Schnitte erstellt und gespeichert.

% !TeX root = ../Notizen.tex

\section*{Aufgabe 1: Bifurkationsdiagramme}
	In dieser Aufgabe, werden die Bifurkationsdiagramme für zwei unterschiedliche Abbildungen erstellt.
	\begin{align}
		&\text{logistische Abbildung: }x_{n+1}=rx_n(1-x_n)\text{ wobei gilt }x_n\in[0,1]\label{eq:logistik}\\
		&\text{kubische Abbildung: }x_{n+1}=rx_n-x^3_n\text{ wobei gilt }x_n\in[-\sqrt{1+r},\sqrt{1+r}]
	\end{align}
	Durch diese Iteration lassen sich Fixpunkte $f(x^*)=x^*$ finden.
	Für $r>0$ gibt es nur einen Fixpunkt $x^*=0$
	und für $r>1$ergibt sich ein weiterer Punkt. 
	Interessant wird das dieses Verfahren für große $r$ keine Fixpunkte mehr findet sondern nur noch Orbits, auf denen um den Fixpunkt oszilliert wird.
	Ein Beispiel ist in \cref{fig:Orbit}.
	Mit weiter anwachsenden $r$ nimmt die zahl der Orbits immer weiter zu.
	Jeder Orbit $x_n$ ist ein Teil des Bifurkationsdiagramms.
	Dazu wird eine Funktion geschrieben, der eine Anzahl an Iterationen über geben wird, und wie viele Werte von aus Gespeichert werden sollen.
	Dieser Ansatz kann verwendet werden unter der Annahme das die Anzahl der Iterationen groß genug ist, dass sich der Algorithmus schon in der Oszilation befindet
	\begin{figure}
		\centering
		\includegraphics[width = 0.5\textwidth]{../Plots/Aufgabe1a/Plot_70.png}\caption{Ein Beispiel Plot für ein Orbit.}\label{fig:Orbit}
	\end{figure}
	\begin{figure}
		\centering
		\includegraphics[width = \textwidth]{../Plots/Plot_1_a.png}
		\caption{Hier ist das Ergebnis für die logistische Abbildung dargestellt.}
	\end{figure}
	\begin{figure}
		\centering
		\includegraphics[width = \textwidth]{../Plots/Plot_1_b.png}
		\caption{Hier ist das Ergebnis für die kubische Abbildung dargestellt.}
	\end{figure}
	Für die Ergebnisse wurden 1000 Iterationsschritte durchgeführt und die letzten 500 werden gespeichert.
\newpage
% !TeX root = ../Notizen.tex

\section*{Aufgabe 2: Harmonischer Oszillator}
\subsection*{a)}
Hier wird das Programm aus Aufgabe 1 getestet, dazu wird der harmonische Oszillator behandelt
\begin{align}
	\frac{1}{m}\vec{F}(\vec{r})=-\vec{r}.
\end{align}
Die Anfangsbedingungen werden so 
\begin{align}
	\vec{r}_0=10\cdot\vec{e}_x,\ \vec{v}_0=\vec{0}
\end{align}
festgelegt und für eine Breite von $h=0,1$ und für $t\in[0,10]s$.\\
\begin{figure}[h!]
	\centering
	\includegraphics[width = 0.75\textwidth]{../Plots/Plot_2_A_1.pdf}
	\caption{Ergebnis für $x(t)$ für den harmonischen Oszillator.\label{fig:Zentral_X}}
\end{figure}\\
In \cref{fig:Zentral_X} ist die Lösung Dargestellt für $x(t)$, dabei hat sie die Form einer harmonischen Schwingung.\\
\newpage
Als nächstes Wird der Fall betrachtet 
\begin{align}
	\vec{r}_1=10\cdot\vec{e}_x,\ \vec{v}_2=10\cdot\vec{e}_y
\end{align}
\begin{figure}
	\centering
	\includegraphics[width = 0.49\textwidth]{../Plots/Plot_2_A_3}
		\includegraphics[width = 0.49\textwidth]{../Plots/Plot_2_A_2}
		\caption{Lösung für $x(t)$ und ein mal Lösung für $\vec{r}(t)$, für den Harmonischen Oszillator für $\vec{r}_1$ und $\vec{v}_1$. }
\end{figure}
Es lässt sich erkennen das $x(t)$, wieder eine harmonische Schwingung darstellt und $r(t)$ Ergibt eine harmonische Schwingung um den Koordinaten Uhrsprung.
\newpage
\subsection*{b)}
Hier wird überprüft ob in dem simulierten System erhalten ist.
Dazu wird die Energiedifferenz $\Delta E= |E(t_n)-E(t_{n-1})|$ bestimmt.
Dabei gilt für die Energie
\begin{align}
	E = T + V =& \frac{1}{2}m\left( v^2 - \int \vec{F}(\vec{r} d\vec{r}) \right) =\frac{1}{2}m\left( v^2 + r^2 \right) = \text{const}\\
	\Rightarrow \Delta E =& 0
\end{align}
bestimmt.\\
\begin{figure}
	\centering
	\includegraphics[width = \textwidth]{../Plots/Plot_2_B_Energie.pdf}
	\caption{Die Energie Differenz für verschiedene $t$\label{fig:EnergieDiff}}
\end{figure}
In \cref{fig:EnergieDiff} ist das Ergebnis dargestellt, dabei ist drauf zu achten das $\Delta E$ mit dem Faktor $10^{-10}$ skaliert.
Es gilt demnach $\Delta E\approx 0$



\end{document} 
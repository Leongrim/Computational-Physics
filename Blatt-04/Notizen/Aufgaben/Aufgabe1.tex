% !TeX root = Notizen.tex

\section*{Aufgabe1: Doppelpendel}
\subsection*{a)}
Die Orte der Massen $m_1$ und $m_2$ im Doppelpendel sind:
\begin{align}
\vec{r_1}=\begin{pmatrix}\sin(\theta_1)\\\cos(\theta_1)\end{pmatrix}
\cdot\SI{1}{m}~~\text{und}~~
\vec{r_2}=\begin{pmatrix}\sin(\theta_1)+\sin(\theta_2)\\\cos(\theta_1)+\cos(\theta_2)\end{pmatrix}\cdot\SI{1}{m}.
\end{align}
Die Geschwindigkeiten sind also:
\begin{align}
\dot{\vec{r_1}}=\begin{pmatrix}\dot{\theta_1}\cos(\theta_1)\\-\dot\theta_1\sin(\theta_1)\end{pmatrix}\cdot\SI{1}{\meter\per\second}
~~\text{und}~~
\vec{r_2}=\begin{pmatrix}\dot\theta_1\cos(\theta_1)+\dot{\theta_2}\cos(\theta_2)\\\-\dot{\theta_1}sin(\theta_1)-\dot{\theta_2}\sin(\theta_2)\end{pmatrix}\cdot\SI{1}{\meter\per\second}.
\end{align}
Die potentielle Energie $V=m\cdot g\cdot h$ ist
\begin{align}
V=&-(m_1+m_2)gL_1\cos(\theta_1)-m_2gL_2\cos(\theta_2)\\
=&-g(2\cos(\theta_1)+\cos(\theta_2))\cdot\SI{1}{\meter\kilogram}~,
\end{align}
wobei $m_1=m_2$ und $L_1=L_2=\SI{1}{m}$ gilt.\\
Die kinetische Energie $T=\frac{1}{2}m{\dot{\vec{r}}}^2$ ist
\begin{align}
T=\frac{1}{2}\left(2\dot\theta_1^2+\dot\theta_2^2+2\cos(\theta_1-\theta_2)\dot\theta_1\dot\theta_2\right)\cdot\SI{1}{\square\meter\kilogram}~.
\end{align}
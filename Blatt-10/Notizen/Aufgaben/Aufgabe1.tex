% !TeX root = ../Notizen.tex

\section*{Aufgabe1: }
\subsection*{a)}
Die Ausrichtungen eines Moleküls soll zufällig erfolge.
Dabei sollen die Ausrichtung mithilfe von Kugelkoordinaten der Einheitskugel erfolgen, wo alle Punkte gleich wahrscheinlich sind.
\begin{align}
	\begin{pmatrix}
		x\\ y\\ z
	\end{pmatrix}
	=
	\begin{pmatrix}
	\cos\phi\sin\theta\\
	\sin\phi\sin\theta\\
	\cos\theta
	\end{pmatrix}
\end{align}
Wird die Kugel als Schichten von Kreisen gedacht, so lässt sich erkennen das der Umfang der Kreise vom Winkel $\theta $ abhängt.
\begin{align}
	U=2\pi \sin\theta
\end{align}
Daraus lässt sich eine Verteilung für die Winkel schließen
\begin{align}
	P(\theta)=\frac{1}{2}\sin\theta, \text{ für } \theta\in[0,\pi]
\end{align}
Diese Verteilung lässt sich realisieren mithilfe der Transformationsmethode
\begin{align}
	\int\limits_{0}^{\theta_j}\frac{1}{2}\sin\theta d\theta = u_j\\
	\Rightarrow \theta_j =  \cos(-2u_j +1)
\end{align}
wobei $u_j$ eine gleichverteilte Zufallszahl aus 0 und 1 ist.
\begin{figure}
	\centering
	\includegraphics[width = 0.6 \textwidth]{../Plots/Plot_1_A.pdf}
	\caption{Hier ist die Erzeugte Verteilung für $\theta$ dargestellt.}
\end{figure}
\newpage
\subsection*{b)}
Als nächstes wird die Verteilung von
\begin{align}
	\omega(\theta) = \frac{1}{2}(3\cos^2\theta-1)
\end{align}
bestimmt.
Der Mittelwert ist 
\begin{align}
	\left< \omega \right> = 0,00246012.
\end{align}
\begin{figure}[h!]
	\centering
	\includegraphics[width = 0.6\textwidth]{../Plots/Plot_1_B.pdf}
	\caption{Hier ist die Verteilung für $\omega(\theta)$ dargestellt.}
\end{figure}
